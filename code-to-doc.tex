\documentclass[xetex,serif,aspectratio=169]{beamer}
\usepackage{xltxtra}
\usepackage[yyyymmdd]{datetime}
\renewcommand{\dateseparator}{/}


\usepackage{nnt}

\XeTeXlinebreaklocale "th"

\defaultfontfeatures{Scale=1.4}

\title{From Code to Docs: จัดการเอกสารแบบมือโปรด้วย \LaTeX + Git}
\subtitle{เอกสารก็ version control ได้ สวยเป๊ะ เหมือนเขียนโค้ด}
\author{John Appleseed}
\institute[XUT]{มหาวิทยาลัยเทคโนโลยีซักแห่งหนึ่ง (XUT) }
\date{\today}

\begin{document}
\begin{frame}
    \maketitle
\end{frame}

\begin{frame}
    \frametitle{หัวข้อ}
    \tableofcontents
\end{frame}

\section{ทำไมต้อง \LaTeX}
\begin{frame}
    \frametitle{ทำไมต้อง \LaTeX}
\end{frame}

\section{เตรียมเครื่องมือ}
\begin{frame}
    \frametitle{เตรียมเครื่องมือ}
\end{frame}

\section{โครงสร้างไฟล์และการ Compile}
\begin{frame}
    \frametitle{โครงสร้างไฟล์และการ Compile}
\end{frame}

\section{การใช้งานเบื้องต้น}
\begin{frame}
    \frametitle{การใช้งานเบื้องต้น}
\end{frame}

\section{ภาษาไทยใน \LaTeX}
\begin{frame}
    \frametitle{ภาษาไทยใน \LaTeX}
\end{frame}

\section{โค้ด \LaTeX เหมือนโค้ดโปรแกรมยังไง?}
\begin{frame}
    \frametitle{โค้ด \LaTeX เหมือนโค้ดโปรแกรมยังไง?}
\end{frame}

\section{การใช้งานร่วมกับ Git}
\begin{frame}
    \frametitle{การใช้งานร่วมกับ Git}
\end{frame}

\section{Demo}
\begin{frame}[c]{ }
    \centering
    \Huge{Demo}
\end{frame}

\section{Q \& A}
\begin{frame}[c]{ }
    \centering
    \Huge{Q \& A}
\end{frame}


\end{document}

